\chapter{结论}
\label{chap:conclusion}

\section{结论}

本文提出了一种基于LTE的人群密度预测算法,其中主要是使用了基于机器学习的GBDT回归算法进行预测。我们通过抽取相邻区域特征和通过自动编码器生成的特征来预测未来人群密度的变化趋势。首先我们通过现有的LTE规范中使用的技术入手收集基站数据,无需用户安装第三方应用即可得到用户的地理位置信息。接着我们分析了GBDT算法和传统的回归算法的异同,从数学角度验证了GBDT算法针对本文的应用场景有着较好的效果。最后我们在监督学习的基础之上引入了自动编码器自学习的特征,对神经网络在本文中的应用场景进行了适当的探索。

在第四章仿真实验当中我们通过仿真实验可以证明对于未来时刻人群密度预测的误差最终会保证在12\%左右,有着较高的准确率,说明本文提出的算法有着较好的实用性。扩展性的我们加入了自动变器自学习的特征,但是实验结果表明并没有显著的提升预测的准确率,我们需要逐步优化自动变器算法中的参数或优化特征选择来提升算法的准确率。

\section{展望}

由于资源和时间上的限制,本文的算法在以下几个方面上还有待提升。

\begin{enumerate}
    \item 在真实的环境中进行实验,验证算法的正确性。
    \item 算法的准确率在预测时间增加后显著下降,所以在预测的准确率上还有待提升。
    \item 在特征提取上目前仅对周围人群的影响作为特征进行了研究,在下一步我们需要加入诸如天气影响、大型活动以及热门区域等特征,进行全方位的分析。
    \item 在没有在真实环境的条件下,我们需要改进仿真软件的人群模拟算法,目前Pedsim软件仅能根据社会关系模型模拟人群移动,下一步我们要对其进行定制增加其他的模拟算法。
    \item 随着LTE定位技术的深入研究,收集到的LTE数据之后的处理算法有待改进。
\end{enumerate}

从上述几点出发我们可以在仿真算法改进、预测算法和真是环境测试等三个方面进行深入的研究。