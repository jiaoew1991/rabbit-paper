\chapter{绪论}
\label{chap:introduction}

\section{研究背景及意义}

电力产业是关系到民生国计的基础产业,电力的安全供应关系到国家安全战略,以及经济社会发展。随着化石能源的不断枯竭以及大气污染的日益加剧,优化能源结构、构建清洁低碳、安全高效的现代能源体系成为国家能源战略的重要方向。

近年来,为了提高供电可靠性、降低网损、改善环境以及支持需求侧响应,分布式可再生能源迅速发展,越来越多的分布式电源(Distributed Generation,DG)如分布式光伏、风电、电动汽车、储能装置以及微网开始大量接入配电网。传统的配电网是一个功率单向流动的无源网络,而大量DG的接入对传统配电网的拓扑结构、运行规程、控制方式和保护配置等都提出了很大的挑战。

从继电保护角度,未来的变化趋势以及面临的挑战主要包括以下几个方面:首先是短路电流方向特征的变化。DG接入后,改变了传统配电网中短路电流的单向性;其次是短路电流水平的变化。大多分布式可再生能源属于逆变器接口型DG\cite{ustun2011central,naderi2016efficient}(Inverter-interfaced DG,IIDG),所能提供的短路电流水平有限,一般仅为额定电流的1.2倍。此外,某些主动配电网具备特殊情况下的微网运行能力,其在联网和孤岛运行模式下的短路电流具有很大差异。这些都给保护的整定与配合带来了极大的困难;第三,主动配电网具备故障后重构以及正常运行时的优化重构能力,而网络重构会改变拓扑结构,保护必须能够做到迅速适应;第四,在主动配电网中,母线上有源线路增多,进一步增大了配置专用母线保护的成本;最后,仍需要把经济性作为重要约束条件,在设计保护方案时应充分考虑投资成本和运维成本。

目前应对含DG的配网保护主要措施有:DG在配网故障时立即退出;限制DG在配网中的接入位置和接入容量;在DG支路增加限流器限制DG贡献的故障电流,这些做法虽然不需要改变原有配网保护系统,但一定程度上破坏了DG的正常运行,没有从根本上解决含DG的配网保护问题。尚处于理论研究阶段的保护方案中,采用多代理的智能保护设计依赖的信息较多、判断方式较复杂,距离实际运用较远;改变原有的电流保护方案,引入方向启动判据是一种整体效果较好的解决方案\cite{shangjin2013}。

现有变流器仿真模型以稳态和机电暂态过程仿真为主,不能准确反映故障电流的时域特征,不能适应保护原理的研究;虽然有模型在电力电子器件部分采用了详细模型,但控制策略较为简单,仅考虑了稳态运行时控制的基本要求,未能计及各种控制策略(如正序、负序控制)、及变流器电流限制;导致不能满足并网规程中故障穿越的要求,也无法兼顾不对称故障的情况,不能适应保护原理的研究。为此,有必要针对继电保护的需要,建立变流器接口型分布式电源的通用故障分析模型,形成实用化的短路电流计算方法。

传统的方向元件通过比较故障电流和参考电压的相位来获得故障方向,需要同时安装电流互感器和电压互感器。这对于线路数量庞大的配电网是不可接受的,一方面增设电压互感器的成本过高;另一方面,此类方向元件存在“近区故障”,即当故障点与保护安装处距离过近时,无法判断故障方向。文献\cite{gao2006design}改进了传统方向性保护的灵敏性和可靠性,用比幅替代比相,解决了因测量电压或电流太小而导致无法判断故障方向的问题,但仍然需要安装电压互感器。仅需电流信息的方向元件不仅可以免除安装电压互感器的成本,还能够解决“近区故障”问题\cite{pradhan2008solution}。因此,研究适用于含DG配电网的方向元件很有必要。

国外学者虽然在有关自适应保护相关的讨论进行了很多年,但实际应用依旧较少。文献\cite{laaksonen2014adaptive}在芬兰的Hailuoto Island上实现了一套自适应的保护与微网控制系统,并投入了实际运行。该系统设置了一个中央控制单元,能够实时检测配电网是否运行与并网或孤岛状态,并根据运行状态的不同切换保护定值,从而实现自适应。但该系统所关注的重点(并网/孤岛运行模式的自适应)与本项目(对配电网结构的自适应,对DG投入/退出的自适应,以及提高保护的选择性和速动性等)不同。

我国学者也对自适应保护做了较多研究,并形成了很多重要的理论成果\cite{yaozhong2007}。但是,相关研究主要针对高压输电网保护。由于传统配电网拓扑结构较为单一、保护配置较为简单,针对配电网的自适应保护研究很少,且缺乏工程应用。


\section{有源配电网对保护系统的影响}

\subsection{有源配电网的发展趋势及对保护的挑战}

随着大量分布式电源(Distributed Generator, DG)接入配电网,传统的单向供电的被动式配电网正在向着能够可靠地完成DG 的消纳、调度、保护和监控的主动配电网方向演变\cite{tianming2013,lipeng2009}。在具有高渗透率DG的主动配电网中,配电网的潮流、短路电流特征将产生实质性的改变\cite{zhang2016new,manditereza2016,huang2016diagnostic},继电保护技术将面临多方面挑战。

第一,在传统城市配电网中的保护一般按照闭环运行来设计,但是实际运行时,配电网是开环运行的,这样保护的配置能够简化。但对于高渗透 DG 的智能配电网,故障电流的单向特性不再存在。配电网采用闭环方式运行可以显著提高供电可靠性、确保线路的电压水平,因此,未来的城市配电网应该有闭环运行能力。然而,在传统的配电网三相电流保护没有方向元件或没有电压互感器配置,很难适应上述趋势。

第二,与常规集中式发电不同的是,DG具有容量小、数量多的特点,运行人员不具备对DG单元的直接控制能力,这意味着配电网的电能质量甚至运行的安全性都难以得到保障。为此,应考虑在DG接入配电网的公共耦合点配置一种接口保护,以使DG能够满足相关并网导则,并简化配电网保护与DG自身保护的协调配合。

第三,DG多点接入配电网后,在进行配电网整定计算时,需考虑系统电源以及DG的多种组合运行方式。这无疑极大增加了短路计算的工作量;由于配电网线路一般较短,保护上下级之间很难配合。这样,整定计算变得非常复杂。仅靠定值和时间的配合已难以保证保护的选择性、灵敏性和速动性。而合理引入光纤通信技术,利用通信来保证保护性能并简化整定计算,成为一种重要的技术趋势。而IEC 61850在高压变电站中的大量成功应用,为实施新型城市配电网保护提供了互操作保证。

第四,很多DG(如风、光、储等)以变流器接入配电网,其故障电流水平与传统的同步发电机存在很大差异,其故障特征从传统的电压源变为受控的电流源。为了满足并网导则的要求,在系统发生不对称故障时,变流器会主动向配电网注入负序电流以提高公共耦合点电压的对称性,同时减少输出功率中的二次脉动分量。这使得变流器接口型DG的相、序故障特征发生了很大变化,给以同步电机故障特征为理论基础的传统保护原理带来了很大挑战。

为了应对上述问题,近年来研究者做出了大量努力\cite{ruisheng2015yi,houlei2014,ruisheng2015shi,liukai2014zhu,nikolaidis2016communication,libin2010han,peichao2016zhi}。由于差动保护可以很好地解决输电网络中双向潮流的问题\cite{ruisheng2015yi},文献\cite{houlei2014}把差动保护应用于配电网,采用基于正序故障分量的电流差动保护原理,有效适应配电网线路特点并能够解决弱馈问题。文献\cite{ruisheng2015shi}采用基于配电自动化的集中式线路差动和就地式母线差动的保护方案,能够适用于主动配电网。但差动保护对通道和同步要求很高\cite{liukai2014zhu},投资成本大、运维复杂,在配电网中大量采用具有困难。

\subsection{适用于有源配电网的保护研究目标}

由上可见,随着DG在城市配电网中的高密度接入,将使得城市电网的电源、负荷及供电特征发生重大变化。为了使得城市配电网能够接纳大量DG,应抓紧研究、示范能够适应上述变化的新型有源配电网自适应保护关键技术。相关关键技术研究与示范应达到如下目的:

第一,自适应性。新型保护方案应能适应配电网的辐射型或环型供电拓扑结构;能够适应DG的随机性以及投入/退出等多种运行方式;能够适应配电网中线路、母线等各种故障位置和故障类型,通过自适应能力,实现全线速动。

第二,经济性。采用一体化、多功能保护装置,减少装置数量;避免采用差动保护来保证绝对的选择性,以降低保护部署和运维的成本;研究无需PT的方向元件。

第三,可靠性。保护原理和方案应能减少保护定值、降低保护上下级配合的复杂性,减少因整定配合所导致的保护不正确动作;充分考虑保护拒动、断路器失灵、通信失效等异常情况,并提供完善的后备保护功能

\section{主要内容与技术路线}

本文首先介绍了配电网中分布式发电并网保护的定义和必要性,分析了其应具备的功能及相应的性能要求。由于并网保护功能的多样性和配置的复杂性,提出采用机器学习的方法实现其中的故障检测和孤岛检测功能。通过理论比较分析,本文采用了泛化能力更优的SVM算法,在传统的二分类基础上,经过概率建模构建并网保护的三分类模型,通过仿真实现并与常规的保护性能进行了比较。最后,为了解决智能型保护实际应用中可能存在的概念漂移现象,针对并网保护中的孤岛检测功能,研究了在线自学习的实现方法。本文将通过六个章节对上述内容展开研究,各章内容如下:

第一章:绪论。分析了含有DG的配电网中保护配置和整定所面临的困境,及采用机器学习方法实现保护原理的优势,并从故障保护和防孤岛保护两方面介绍了国内外已提出的智能保护原理,说明了机器学习方法在继电保护领域的研究现状。

第二章:分布式发电并网保护。梳理了国内外对DG并网保护的研究现状;综合多个标准规程,给出了并网保护的定义,并从故障检测、孤岛检测和检同期等方面分析了并网保护应具备的保护功能及基于常规保护的配置方案。然后针对其中的故障检测和孤岛检测功能分析了相应的性能要求。最后,从并网标准和保护原理两方面讨论了以后可能的研究方向。

第三章:机器学习基本理论与方法。针对并网保护配置复杂、整定困难等问题,本文提出了基于机器学习的解决思路。为此,本章专门对机器学习方法及常用的分类算法作了介绍。通过分析SVM算法的基本理论,比较得出了其理论上的优势,因此确定了本文采用的基本分类算法。最后说明了基于机器学习的智能保护原理的特有优势及目前还待解决的一些问题。

第四章:基于多分类支持向量机的分布式发电并网保护。为了同时实现DG并网保护中的故障检测和孤岛检测功能,本章在传统二分类SVM的基础上进行改进,通过概率建模的方法构建三分类模型,同时能够给出预测类别的概率估计。采用了SVM-RFE特征选择算法结合交叉验证,以提高分类模型的泛化能力。最后,通过对变流器接口型DG和同步电机型DG并网保护的算例进行仿真证明了该方案比常规保护性能上更加可靠。

第五章:主动配电网中基于在线自学习的孤岛检测方法。针对主动配电网中在线孤岛检测时易出现的概念漂移现象,本章提出了在线聚类抽样和优选样本集相结合的在线学习方法。另外,针对实际电网运行样本中孤岛类样本远远少于非孤岛类样本,而导致样本集类别分布不均衡的问题,本章采用了加权SVM算法结合新的评价指标以减弱非均衡数据集的影响。最后通过仿真分别测试了在线聚类抽样和优选样本集策略,证明了所提方案的可行性和合理性。

第六章:总结全文,介绍本文取得的工作成果和课题后续展望。
